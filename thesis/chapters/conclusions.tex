\chapter{Conclusioni}
In quest'ultimo capitolo si traggono le conclusioni sui risultati delle analisi svolte e
sui possibili sviluppi futuri di questo lavoro.

\label{chap:chap6}
\section{Risultati ottenuti}
Si è arrivati a realizzare un sistema di controllo avanzato che permette di 
ottimizzare l'attività del \textit{path tracking} attraverso l'implementazione
in \textit{Python} dell'algoritmo \textit{Model Predictive Control}.
Si è dunque in grado di seguire la linea di riferimento, ovvero la raceline generata dal \textit{planner}, con una 
traiettoria ottimale calcolata in tempo reale lungo l'intero percorso. Ciò permette di 
minimizzare la deviazione dalla linea di riferimento, che è stata misurata mediante il 
\textit{Crosstrack Error} e, inoltre, nelle diverse configurazioni si è osservata spesso una
riduzione dei tempi di percorrenza sul giro.

Sulla base dei risultati presentati nel Capitolo~\ref{chap:chap5}, le diverse configurazioni di
\textit{MPC} risultano superiori a un metodo di controllo geometrico come il 
\textit{Pure Pursuit}, appartenente alla categoria dei metodi reattivi per il controllo di veicoli. 
Si tratta di un risultato previsto, che risulta coerente con la formulazione di \textit{MPC}, 
ovvero come un problema di ottimizzazione con vincoli che è caratterizzato da una struttura 
non banale che guarda al \textit{``futuro''}. 

D'altro canto, il \textit{Pure Pursuit} è un metodo di \textit{path tracking geometrico}, poiché 
calcola l'angolo di sterzata da applicare alle ruote per raggiungere il waypoint successivo nella linea di riferimento. 
Questo metodo, a differenza di \textit{MPC}, presenta un andamento più irregolare; infatti, 
porta a seguire una traiettoria che, osservando i risultati del \textit{Crosstrack Error},
si discosta di molto dalla linea teorica, con valori medi superiori tra il 140\% e il 153\%
rispetto al profilo \textit{High Performance MPC}.
Si è anche rilevato che \textit{Pure Pursuit} consuma molta più energia di ogni metodo di 
\textit{MPC}. Inoltre, come qualsiasi altro metodo di controllo reattivo, esso non considera in
alcun modo la dinamica del sistema, pertanto può produrre archi impraticabili; contrariamente, 
\textit{MPC} incorpora un modello della dinamica, come il \textit{Kinematic Bicycle Model}, discusso nella sezione~\ref{subs:kinmodel}.

Infine, dalle diverse configurazioni di \textit{MPC} emergono risultati differenti tra loro:
la configurazione \textit{High Performance} risulta essere la migliore in un contesto di
guida autonoma con un singolo agente che corre a velocità estreme, prossime ai limiti fisici del 
veicolo. Nello specifico, per questo profilo si ottiene che, per entrambi i circuiti, si
hanno dei \textit{lap time} e degli \textit{RMSE} inferiori, con un buon compromesso per
ciò che riguarda il \textit{consumo energetico}, il quale risulta però minore per i metodi 
\textit{Safe} e \textit{Fast}. Ciò non è casuale, infatti \textit{High Performance} presenta 
prestazioni superiori anche in termini di velocità su entrambe le piste testate. Invece, per 
quanto riguarda l'angolo di sterzata applicato, non si rilevano particolari miglioramenti per i profili di \textit{MPC}.

\section{Sviluppi futuri}
Gli sviluppi futuri per questo progetto possono muoversi verso nuove prospettive,
sia per poterlo applicare per attività più complesse, sia per migliorare la soluzione
di \textit{MPC} realizzata.

\paragraph{Sim2Real} La prima evoluzione consiste nel passaggio dalla simulazione alla realtà. 
Ciò implicherebbe sfide non banali, come l'adattamento di diversi valori di configurazione e di 
certe strategie decisionali a livello implementativo, oltre alla costruzione del veicolo.
In un ambiente reale vi è incertezza: il modello utilizzato nella simulazione è solo 
un'approssimazione e, in più, l'attuazione su un veicolo reale non è più solo prodotta da un software, ma è in larga parte meccanica.
Sarà dunque cruciale effettuare un'attenta attività di \textit{tuning} per ottimizzare le prestazioni.
\paragraph{Modelli più complessi} Si possono adottare modelli più aderenti alla realtà, 
come il \textit{Dynamic Bicycle Model}, che considerano dinamiche non lineari e fenomeni
aerodinamici tipici nell'\textit{autonomous racing} a velocità elevate. 
I modelli non lineari, infatti, potrebbero migliorare la precisione del controllo, al costo 
però di tempi di risoluzione più lunghi e di possibili valori inferiori per la velocità.
Si avrebbe così un problema non convesso, che andrebbe risolto con un risolutore non lineare come \textit{Casadi}.
\paragraph{Competizione multi-agente} Un'altra direzione di ricerca potrebbe
essere data da un contesto di competizione con due (o più) veicoli. 
Questo tipo di lavoro richiederebbe nuove strategie per gestire il comportamento competitivo,
legate alla \textit{teoria dei giochi}. Ciò implicherebbe, ad esempio, lo sviluppo di tecniche 
per ottimizzare i sorpassi, la difesa della traiettoria e la gestione delle collisioni.
\paragraph{Reti Neurali e MPC Data-Driven} Si potrebbero integrare anche delle reti neurali \cite{TATULEACODREAN20206031, fuchs2021super} per
migliorare ulteriormente il processo di \textit{path tracking}. 
Nello specifico, si potrebbero utilizzare tecniche di \textit{Imitation Learning (IL)} e 
\textit{Reinforcement Learning (RL)} al fine di apprendere comportamenti ottimali dai dati registrati nei giri precedenti.
I lavori più recenti sui controller nell'\textit{autonomous racing} si sono concentrati
proprio sullo sviluppo di componenti interni basati sull'apprendimento, come soluzioni di 
\textit{Learning MPC} \cite{xue2024learning, rosolia2019learning} che applicano proprio queste idee.
In particolare, si potrebbe esplorare un approccio ibrido in cui il \textit{controller}
sfrutta tecniche basate su modelli classici e le combina coi dati raccolti da
esperienze passate per aggiornare e migliorare le prestazioni del sistema in tempo reale.