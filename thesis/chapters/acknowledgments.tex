%\newgeometry{top=3.5cm,margin=2.5cm}
\newgeometry{bottom=2.5cm}
\chapter*{Ringraziamenti}
\addcontentsline{toc}{chapter}{Ringraziamenti}
\label{chap:ack}
Mi trovo a scrivere queste parole dopo un percorso molto tormentato, pieno di ostacoli e delusioni.
Queste non riguardano esclusivamente l'esperienza universitaria, ma sono dovute in larga parte a ciò
che ho passato al di fuori di essa in questi quattro anni. 
Tutto questo ha causato stress, tristezza e ritardi, specialmente nel primo semestre del secondo anno, 
che è stato, senza alcun dubbio, il periodo più duro di questi anni.\\
\noindent Innanzitutto, vorrei però ringraziare il mio relatore, il \textbf{Prof. Matteo Luperto}, che ha permesso di
farmi lavorare a un progetto di tesi molto stimolante su tematiche attuali e non banali, ed è sempre
stato disponibile per rispondere ai miei dubbi con ricevimenti e messaggi. \\ 

\noindent Voglio dedicare le prossime righe alle persone più importanti della mia vita:
senza di voi sarebbe stato tutto più complicato. \\

\noindent Ringrazio di cuore \textbf{Rossella}, la mia ragazza.
Ci siamo conosciuti proprio durante quel periodaccio al secondo anno: sei arrivata nel momento del bisogno, e ti sarò sempre immensamente grato per questo. \\
\noindent Grazie per tutto quello che hai fatto per me in questi anni: il tuo sostegno e il tuo amore sono 
stati fondamentali per uscire da quel tunnel pieno di negatività. 
Mi hai aiutato a dare sempre il massimo, senza lasciarmi abbattere.\\ % DA PENSARCI MEGLIO
\noindent Grazie per aver percorso, innumerevoli volte, chilometri su chilometri per vedermi,
anche quando si riusciva solo per poco tempo. Ogni attimo passato con te è stato linfa vitale; sei 
sempre riuscita a farmi distrarre dallo studio quando ne avevo bisogno.
Infatti, voglio anche ringraziarti per avermi sopportato in quei periodi di maggior stress; 
per avermi consolato quando un esame non andava bene; per aver letto ogni messaggio di sfogo; per aver 
ascoltato ogni audio di almeno due minuti o un intero monologo in videochiamata riguardante ciò che 
stavo studiando o che restava ancora da fare.
Grazie per aver provato a capire ciò che ho studiato in questi anni. Non l'hai fatto perché ti 
sentivi obbligata, ma perché volevi darmi, da buona \textit{perfezionista}, dei consigli consapevoli.\\
\textbf{Ross}, grazie per tutto ciò che sei. \textbf{Sei speciale.} \\

\noindent Voglio poi ringraziare i miei \textbf{genitori}.
Mi risulta difficile descrivere in poche parole ciò che provo per voi, solo io sono a conoscenza dei \textit{sacrifici} che avete fatto per me in questi 23 anni.
Grazie per avermi permesso di arrivare a questo momento e scusatemi per tutte le volte che sono stato intrattabile durante le sessioni. \\

\noindent Inoltre, voglio ringraziare profondamente le persone che ho conosciuto in questo percorso
di studi: ci siamo aiutati a vicenda su Discord per la preparazione di ogni esame, abbiamo passato
insieme tanti momenti stupendi, ma anche altri negativi. Infatti, si sono venuti a creare dei legami
forti, e sono sicuro che senza il vostro aiuto sarebbe stato tutto più noioso e intricato.
Siete stati importanti e desidero ringraziarvi uno ad uno: \textbf{Emanuele 
Manca\footnote{\href{https://youtu.be/3-4u6BVLHcI}{https://youtu.be/3-4u6BVLHcI} $\leftarrow$ Qui potete vedere me ed Ema dopo aver dato l'ultimo esame della triennale.}} (\textit{Ema}),
\textbf{Davide Papasodaro} (\textit{Papas}), \textbf{Gabriele Giorgio} (\textit{Gabri}), 
\textbf{Federico Marcelli Fabiani} (\textit{Fede}), \textbf{Marco Morandi} (il fu \textit{Jimmy}), 
\textbf{Ivan} (\textit{il}) \textbf{Selvaggio}, \textbf{Gabriele Gilberti} (\textit{Gigi}),\\
\textbf{Cristian Pozzi} (\textit{Criii Pòtzieh}), \textbf{Federico Coscia} (\textit{Fedoscia}). 
Ultimo, ma non per importanza, \textbf{Navjot Kumar} (\textit{Nav}), il mio vecchio amico d'infanzia ed 
ex compagno delle superiori -- assieme a \textbf{Ema} e \textbf{Cri} -- col quale ho condiviso anche il primo anno.
%Voglio ringraziare anche altri ragazzi che, pur non facendo parte del mio gruppo più stretto, sono 
Voglio ringraziare anche \textbf{Mattia Oldani} che, pur non facendo parte del mio gruppo più stretto, è
stato d'aiuto durante l'ultimo anno. Noi non dimenticheremo quei parziali di Fisica.\\

\noindent Infine, voglio ringraziare tutti i miei \textbf{amici} più cari di Vercelli, in particolar modo 
\textbf{Mattia Manzoni} (\textit{{\Huge $\mathbb{B}$}attia!}), il mio miglior amico, fresco fresco di 
certificazione per personal trainer. Vi ringrazio tutti perché nei weekend di questi ultimi anni mi avete 
aiutato a rilassarmi e distrarmi, facendomi passare delle belle serate in compagnia. 
Mi dispiace per tutti quei ``Non ci sono, mi devo svegliare presto domani'', ``Non riesco, devo ripassare per l'esame'' e così via.
\vskip 1cm
\raggedleft \textit{Ne è valsa la pena.}