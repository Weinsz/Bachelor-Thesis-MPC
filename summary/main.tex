\documentclass[12pt,a4paper]{article}
\usepackage[margin=2.5cm]{geometry}
%\usepackage[a4paper,top=2cm,bottom=2cm,left=3cm,right=3cm]{geometry}
%\usepackage[a4paper,top=2cm,bottom=2cm,left=3cm,right=3cm]{geometry}
\usepackage[italian]{babel}			% Supporto per l'italiano
\usepackage[a-1b]{pdfx}			    % File conforme allo standard PDF-A (obbligatorio per la consegna)
%\usepackage[colorlinks=true, allcolors=blue]{hyperref}
%\usepackage{parskip}			    
\title{\textbf{Ottimizzazione del Path Tracking con Model Predictive Control per Autonomous Racing in F1TENTH}}
\author{\Large Vincenzo Siano, matricola 981734}

\begin{document}
\date{}
\maketitle
Il settore della guida autonoma ha registrato notevoli progressi negli 
ultimi anni, favorito dall'innovazione portata in quest'ambito dalla robotica e dall'intelligenza artificiale. Il path tracking rappresenta una delle sfide principali per questi veicoli, specialmente in contesti competitivi
come l'Autonomous Racing, nel quale la precisione e le performance sono cruciali. 
La comunità di F1TENTH offre una piattaforma che permette di testare i 
propri algoritmi su veicoli autonomi in scala 1:10, simulando corse reali in un ambiente controllato e adatto a valutare le prestazioni delle soluzioni sviluppate.

\medskip
\noindent Questa tesi si propone di sviluppare e di analizzare un sistema di controllo ottimizzato per l'attività 
di path tracking con un singolo agente, applicata al campo dell'Autonomous Racing per la piattaforma 
F1TENTH, mediante l'implementazione di un algoritmo di controllo ottimale e predittivo, il Model Predictive Control (MPC).
Inoltre, tale sistema è stato confrontato con Pure Pursuit, un altro metodo di controllo più semplice 
a livello concettuale. Nello specifico, gli obiettivi del lavoro comprendono:
\begin{enumerate}
    \item lo sviluppo di un sistema funzionante di guida autonoma di un veicolo simulato della categoria F1TENTH;
    \item l'ottimizzazione della traiettoria calcolata, riducendo il più possibile l'errore di 
    tracking, ovvero la deviazione rispetto alla linea teorica di riferimento;
    \item l'applicabilità in circuiti di Formula 1 in scala ridotta a 1:10;
    \item l'analisi dei risultati e il confronto delle prestazioni col Pure Pursuit e tra le diverse configurazioni di MPC. 
\end{enumerate}
Pertanto, lo scopo ultimo è comprendere se Model Predictive Control rappresenta una soluzione valida e se sia superiore
a un metodo di controllo reattivo più semplice, siccome effettua calcoli puramente geometrici per
seguire una traiettoria calcolata da un planner.

\section{Descrizione del lavoro}
In primo luogo, è stato effettuato uno studio preliminare della piattaforma F1TENTH e dei concetti 
fondamentali della guida autonoma, ponendo particolare attenzione al controllo del veicolo in 
contesti di corse con un singolo agente.
Si è poi proseguito con lo studio di Model Predictive Control, organizzando i parametri necessari per la configurazione dell'algoritmo in file appositi.
Successivamente è stato implementato l'algoritmo, prima attraverso l'impostazione del problema di ottimizzazione convessa, secondo un risolutore adeguato; poi, attraverso la definizione della funzione obiettivo e dei vincoli, è stato ottimizzato il controllo del veicolo lungo la traiettoria di riferimento. 

\medskip
\noindent 
Alla scrittura del codice è seguita una fase di ottimizzazione (tuning) delle matrici di 
pesi presenti nella funzione obiettivo di MPC, volta a migliorare la precisione nel path tracking 
senza però compromettere la stabilità e la velocità del veicolo. \\
I test sono stati condotti su due circuiti di Formula 1, Spa-Francorchamps e Monza, mediante l'uso del simulatore di F1TENTH. Da questa fase, sono emersi tre profili di guida, ciascuno ottimizzato per scenari differenti.
Durante le simulazioni, sono stati raccolti i dati relativi alla posizione, all'angolo di sterzata, alla velocità e all'accelerazione. A partire da questi dati sono state
calcolate specifiche metriche come il Crosstrack Error, al fine di valutare le performance del sistema. 

\medskip
\noindent In ultima istanza, i risultati ottenuti sono stati analizzati attraverso dei notebook di Jupyter, all'interno dei quali sono stati effettuati confronti tra MPC e Pure Pursuit, oltre a valutare l'efficacia dei tre profili di MPC.
\section{Tecnologie coinvolte}
Il lavoro ha coinvolto le seguenti tecnologie:
\begin{itemize}
    \item ROS 2 (Robot Operating System), per la gestione dell'infrastruttura robotica;
    \item F1TENTH Gym, per la simulazione dei circuiti e del veicolo. Tuttavia, l'effettiva visualizzazione del simulatore avviene su RViz;
    \item Python, come linguaggio per l'implementazione degli algoritmi. Sono state adottate librerie come rclpy 
    per l'interazione con ROS 2; CVXPY, necessaria per la risoluzione del problema di ottimizzazione convessa; Jupyter Notebook, invece, per l'analisi dei risultati.
\end{itemize}

\section{Competenze e risultati raggiunti}
%- Quali risultati sono stati raggiunti rispetto agli obiettivi iniziali?
%- Quali insegnamenti si possono trarre dall’esperienza effettuata?
%- Quali i problemi incontrati? Quali risolti e quali no? Perch´e?
L'analisi dei risultati ha confermato l'ottima funzionalità di Model Predictive Control su entrambi
i circuiti testati, in linea con le aspettative e gli obiettivi prefissati. Grazie alla sua natura 
predittiva, MPC fornisce traiettorie più precise, con errori di tracking significativamente inferiori 
rispetto al Pure Pursuit. Inoltre, si osservano anche lap time inferiori per ogni profilo di MPC 
con la configurazione ``High Performance'' che si è rivelata come la migliore, soprattutto in uno 
scenario di gara estrema con un singolo agente, garantendo prestazioni superiori anche in termini di velocità.
Un ulteriore aspetto positivo è emerso dall'analisi dei consumi energetici, aspetto nel quale MPC ha mostrato 
un'efficienza superiore rispetto al Pure Pursuit, un risultato particolarmente rilevante data la 
scala ridotta del veicolo.
\end{document}